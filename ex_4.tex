\documentclass[11pt]{article}

\usepackage{amsmath}
\usepackage{amsfonts}
\usepackage{amssymb}
\usepackage{amsthm}

\marginparwidth 0pt
\oddsidemargin  0pt
\evensidemargin 0pt
\marginparsep   0pt
\topmargin     -40pt
\textwidth      6.5 in
\textheight     9 in

\usepackage{natbib}
\bibpunct{(}{)}{;}{a}{}{,} % to follow the A&A style
\bibliographystyle{aa}

\usepackage{physics}

\begin{document}

\noindent

{\bf Exercise 4}: Imagine a set of N measurements $t_i$, with uncertainty variances $\sigma^2_{t_i}$, all of the same (unknown) quantity T. Assuming the generative model that each $t_i$ differs from T by a Gaussian-distributed offset, taken from a Gaussian with zero mean and variance $\sigma^2_{t_i}$, write down an expression for the log likelihood ln L for the data given the model parameter T. Take a derivate to show that the maximum likelihood value for T is the usual weighted mean.

{\bf Solution:}
The frequency distribution for the data $y_i$ in this model is
\begin{equation}
    p(t_i | \sigma_t_i, T) = \frac{1}{\sqrt{2 \pi \sigma_t_i^2}} \exp\bigg(-\frac{(t_i - T)^2}{2 \sigma_t_i^2} \bigg) ,
\end{equation}

The objective function that maximizes the probability of the observed data given the model is the likelihood, defined as
\begin{equation}
    \mathfrak{L} = \prod_{i=1}^{N} p(t_i | \sigma_t_i, T)
\end{equation}
since the data are assumed to be independent. Taking the logarithm,
\begin{equation}
    \ln{\mathfrak{L}} = C - \sum_{i=1}^{N} \frac{(t_i - T)^2}{2\sigma_t_i^2}
\end{equation}
for some constant C. Maximizing $\ln{\mathfrac{L}}$ by taking the derivative with respect to the model parameter $T$ gives
\begin{equation}
    0 = \pdv{ln{\mathfrak{L}}}{T} = + \sum_{i=1}^{N} \frac{t_i - T}{\sigma_t_i^2}
\end{equation}
\begin{equation}
    \implies NT = \sum_{i=1}^{N} t_i
\end{equation}
\begin{equation}
    \implies T = \frac{1}{N} \sum_{i=1}^{N} t_i
\end{equation}
This is the usual weighted mean, $\bar{t} = \frac{\sum_{i=1}^{N} w_i t_i}{\sum_{i=1}^{N} w_i}$ with weights $w_i = 1$.

\end{document}
