\documentclass[11pt]{article}

\usepackage{amsmath}
\usepackage{amsfonts}
\usepackage{amssymb}
\usepackage{amsthm}

\marginparwidth 0pt
\oddsidemargin  0pt
\evensidemargin 0pt
\marginparsep   0pt
\topmargin     -40pt
\textwidth      6.5 in
\textheight     9 in

\usepackage{natbib}
\bibpunct{(}{)}{;}{a}{}{,} % to follow the A&A style
\bibliographystyle{aa}

\usepackage{physics}

\begin{document}

\noindent

{\bf Exercise 5}: Take the matrix formulation for χ2 given in equation (7) and take derivatives to show that the minimum is at the matrix location given in equation (5).

{\bf Solution:} Equation (7) reads
\begin{equation}
    \chi^2 = \sum_{i=1}^{N} \frac{(y_i - f(x_i))^2}{\sigma_{y_i}^2} \equiv \mathbf{(Y - A X)^T C^{-1} (Y - A X)}
\end{equation}
In order to take derivatives, we will first convert this matrix expression into an equivalent one involving matrix components:
\begin{equation}
    \begin{aligned}
        \chi^2  &= \mathbf{(Y^T - X^T A^T) C^{-1} (Y - A X)} \\
                &= (y_i - x_k A_{ki}) C_{il}^{-1} (y_l - A_{ln} x_n) \\
                &= y_i C_{il}^{-1} y_l - x_k A_{ki} C_{il}^{-1} y_l - y_i C_{il}^{-1} A_{ln} x_n + x_k A_{ki} C_{il}^{-1} A_{ln} x_n
    \end{aligned}
\end{equation}
where the Einstein summation convention has been utilized. Taking the derivative to find the minimum,
\begin{equation}
    \begin{aligned}
        0 = \pdv{\chi^2}{x_m}   &= - A_{mi} C_{il}^{-1} y_l - y_i C_{il}^{-1} A_{lm} + A_{mi} C_{il}^{-1} A_{ln} x_n + x_k A_{ki} C_{il}^{-1} A_{lm} \\
                            &= - \mathbf{A^T C^{-1} Y - Y^{T} C^{-1} A + A^T C^{-1} A X + X^T A^T C^{-1} A}
    \end{aligned}
\end{equation}
Notice that each of these terms is a scalar, and so must equal their transpose. For example, $\mathbf{(A^T C^{-1} y)^T} = \mathbf{y^T C^{-1} A} = \mathbf{A^T C^{-1} y}$ since $\mathbf{C^{-1}}$ is symmetric. The first two and the last two terms are equal, and so
\begin{equation}
    \begin{aligned}
        0 &= - 2 \mathbf{A^T C^{-1} Y} + 2 \mathbf{A^T C^{-1} A X} \\
        \implies \mathbf{X} &= \mathbf{(A^T C^{-1} A)^{-1} (A^T C^{-1} Y)}
    \end{aligned}
\end{equation}
This is exactly the expression given in equation (5) of the text.

\end{document}
